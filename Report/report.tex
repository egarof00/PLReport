\documentclass{article}

\usepackage{tikz} 
\usetikzlibrary{automata, positioning, arrows} 

\usepackage{amsthm}
\usepackage{amsfonts}
\usepackage{amsmath}
\usepackage{amssymb}
\usepackage{fullpage}
\usepackage{color}
\usepackage{parskip}
\usepackage{hyperref}
  \hypersetup{
    colorlinks = true,
    urlcolor = blue,       % color of external links using \href
    linkcolor= blue,       % color of internal links 
    citecolor= blue,       % color of links to bibliography
    filecolor= blue,        % color of file links
    }
    
\usepackage{listings}

\definecolor{dkgreen}{rgb}{0,0.6,0}
\definecolor{gray}{rgb}{0.5,0.5,0.5}
\definecolor{mauve}{rgb}{0.58,0,0.82}

\lstset{frame=tb,
  language=haskell,
  aboveskip=3mm,
  belowskip=3mm,
  showstringspaces=false,
  columns=flexible,
  basicstyle={\small\ttfamily},
  numbers=none,
  numberstyle=\tiny\color{gray},
  keywordstyle=\color{blue},
  commentstyle=\color{dkgreen},
  stringstyle=\color{mauve},
  breaklines=true,
  breakatwhitespace=true,
  tabsize=3
}

\newtheoremstyle{theorem}
  {\topsep}   % ABOVESPACE
  {\topsep}   % BELOWSPACE
  {\itshape\/}  % BODYFONT
  {0pt}       % INDENT (empty value is the same as 0pt)
  {\bfseries} % HEADFONT
  {.}         % HEADPUNCT
  {5pt plus 1pt minus 1pt} % HEADSPACE
  {}          % CUSTOM-HEAD-SPEC
\theoremstyle{theorem} 
   \newtheorem{theorem}{Theorem}[section]
   \newtheorem{corollary}[theorem]{Corollary}
   \newtheorem{lemma}[theorem]{Lemma}
   \newtheorem{proposition}[theorem]{Proposition}
\theoremstyle{definition}
   \newtheorem{definition}[theorem]{Definition}
   \newtheorem{example}[theorem]{Example}
\theoremstyle{remark}    
  \newtheorem{remark}[theorem]{Remark}

\title{CPSC-354 Report}
\author{Emma Garofalo  \\ Chapman University}

\date{\today} 

\begin{document}

\maketitle

\begin{abstract}
This document outlines what has been learned week by week through this class. For now, it only contains the information learned for week one.
\end{abstract}

\setcounter{tocdepth}{3}
\tableofcontents

\section{Introduction}\label{intro}
Welcome to my class report! As the class progresses, I will add my learning
from each week. For now, it only has week one. 
\section{Week by Week}\label{homework}

\subsection{Week 1}
\subsubsection*{Notes}
This week we discussed the foundation of what the class is about.
The main idea was that this class is largely about the intersection
of math and programming, beginning with revisiting the principles we learned
in discrete mathematics. We also learned about LaTeX, which we will be using throughout 
the semester to edit documents like this one. The general idea of the week was setting all of us 
students up for what to expect throughout the semester.

We also covered the topics of Formal Systems, which are explained in more detail in the below
section. And how to determine the relations between a Lean proof and a Math proof.
\subsubsection*{Homework}

The reading that we had to cover for homework discussed the MUI problem, which helps us
break down what a formal system is, and how these attributes can be seen in mathematics like 
discrete math. A formal system carries the requirement of formality, which states that you
must not do anything outside of the set rules. Theorems, axioms, rules of production, and the
decision procedure are all key parts of a formal system.

We also had to complete the tutorial world of the natural number game. 
Here are the solutions for levels 5-8.

\subsubsection*{Level 5}
Goal: a+(b+0)+(c+0)=a+b+c

Solution:
\begin{lstlisting}
rw[add_zero]
rfl
\end{lstlisting}
\subsubsection*{Level 6}
Goal: a+(b+0)+(c+0)=a+b+c

Solution:
\begin{lstlisting}
rw[add_zero c]
rw[add_zero]
rfl
\end{lstlisting}
\subsubsection*{Level 7}
Goal: succ n = n + 1

Solution:
\begin{lstlisting}
rw[one_eq_succ_zero]
rw[add_succ]
rw[add_zero]
rfl
\end{lstlisting}
\subsubsection*{How is this lean proof related to mathematics proofs?:}
This Lean solution demonstrates the definition of Natural Numbers. 
Natural numbers are defined by two principles:
1. "There is a special natural number, called zero, denoted by 0."
2. "For any natural number n, there is a unique next natural number, called
the successor of n."

The first step completed in the Lean proof uses rule 2 by breaking the number
1 on the right side of the equation into the successor of 0.
The next line uses the property of addition for successors. Showing that n + succ(m) = succ(n+m)
The additive identity property is then used to simplify the equation so that both sides are now
equal to each other, proving the validity of the statement that the succ n = n + 1
\subsubsection*{Level 8}
Goal: 2 + 2 = 4

Solution:
\begin{lstlisting}
nth_rewrite 2[two_eq_succ_one]
rw[one_eq_succ_zero]
rw[four_eq_succ_three]
rw[three_eq_succ_two]
nth_rewrite 2[two_eq_succ_one]
rw[one_eq_succ_zero]
rw[two_eq_succ_one]
rw[one_eq_succ_zero]
rw[add_succ]
rw[add_succ]
rw[add_zero]
rfl
\end{lstlisting}

I learned a lot from this homework. It basically acted as a refresher for discrete mathematics, and 
how what seems like such a simple solution is much more complicated than you think it is. It also shows 
the various ways that you can derive the same solutions.


%In case you want to draw automata in Latex, you can use the tikz %package. Here is an example of a simple automaton:
%
%\begin{tikzpicture}[shorten >=1pt,node distance=2cm,on grid,auto] 
%  \node[state] (q_1)   {$q_1$}; 
%  \node[state] (q_2) [above right=of q_1] {$q_2$}; 
%  \node[state] (q_3) [below right=of q_2] {$q_3$}; 
%   \path[->] 
%   (q_1) edge  node {0} (q_2)
%         edge  node [swap] {1} (q_3)
%   (q_2) edge  node  {1} (q_3)
%         edge [loop above] node {0} ()
%   (q_3) edge [loop below] node {0,1} ();
%\end{tikzpicture}
%
%By the way, GPT-4 is quite good at outputting tikz code.

\subsubsection*{Comments and Questions}

This week provided me with a good refresher of the discrete mathematics class that I took a while ago. It brought to my attention how much
there is a crossover between math and code, and I am so excited to explore that in this class.

My question for the week relates to the foundation of mathematics, and where it has all evolved from there. We refreshed on discrete math,
which shows us why even the simplest mathematic proofs are valid. And it makes me wonder how mathematics has evolved so much since then. We have
such calculated math that has all built off of these proofs. Seeing how much math has evolved since then, does that mean that math will continue to
evolve in complexity forever? As we create new technologies and understand our world better, will more complicated relationships continue to be found?
\subsection{\ldots}

\ldots

% \section{Lessons from the Assignments}

% (Delete and Replace): Write three pages about your individual contributions to the project.

% On 3 pages you describe lessons you learned from the project. Be as technical and detailed as possible. Particularly valuable are \emph{interesting} examples where you connect concrete technical details with \emph{interesting} general observations or where the theory discussed in the lectures helped with the design or implementation of the project.

% Write this section during the semester. This is approximately a quarter of apage per week and the material should come from the work you do anyway. Just keep your eyes open for interesting lessons.

% Make sure that you use \LaTeX{} to structure your writing (eg by using subsections).

% \section{Conclusion}\label{conclusion}

% (Delete and Replace): (approx 400 words) A critical reflection on the content of the course. Step back from the technical details. How does the course fit into the wider world of software engineering? What did you find most interesting or useful? What improvements would you suggest?

% \begin{thebibliography}{99}
% \bibitem[BLA]{bla} Author, \href{https://en.wikipedia.org/wiki/LaTeX}{Title}, Publisher, Year.
% \end{thebibliography}

\end{document}
